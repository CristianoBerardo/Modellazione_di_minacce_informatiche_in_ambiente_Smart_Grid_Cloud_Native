\afterpage{\blankpage}
\newpage
% \chapter{Threat Model - Modellazione delle minacce}
% \chapter{Modellazione di minacce informatiche}

% \chapter{Minacce informatiche: analisi, tecniche e identificazione}

\chapter{Modellazione delle Minacce Informatiche: Metodologie e Framework Applicativi}


% who will attack the system, how the
% attack will be deployed, and outlining possible security measures and controls to
% stop the threat before it damages the system.


% Nel capitolo precedente ho introdotto i benefici dell'utilizzo del Cloud Computing, arrivando a presentare una Smart Grid in un contesto Cloud-native decentralizzato con la sua descrizione.

% In questo capitolo invece mi occuperò del fulcro di questa ricerca di tesi, ovvero la modellazioni di minacce informatiche (threat modelling) in questo ambiente Smart Grid cloudificato.

Nel capitolo precedente è stata presentata un'architettura per la Smart Grid basata su un paradigma \textit{Cloud-Native} decentralizzato. È stato dimostrato come tale approccio offra significativi vantaggi in termini di scalabilità, resilienza e agilità. Tuttavia, la transizione da sistemi on-premise a infrastrutture distribuite e basate su cloud altera profondamente il panorama delle minacce, introducendo nuove superfici di attacco e vettori di compromissione.


Pertanto, una valutazione proattiva e sistematica di queste nuove vulnerabilità diventa un'attività indispensabile per garantire la sicurezza e l'affidabilità dell'intera infrastruttura. Questo capitolo costituisce il nucleo metodologico della tesi, fornendo gli strumenti concettuali per la modellazione delle minacce (\textit{Threat Modeling}), un processo strutturato per identificare, analizzare e mitigare i rischi di sicurezza fin dalla fase di progettazione di un sistema.

La trattazione seguirà un percorso logico:
\begin{enumerate}
    \item Inizialmente, verrà definito il processo di \textit{Threat Modeling} e i suoi ambiti applicativi.
    \item Successivamente, verranno illustrati i passi fondamentali che compongono questa metodologia.
    \item Verrà poi introdotto in dettaglio il framework STRIDE, la metodologia scelta per la classificazione sistematica delle minacce in questo studio.
    \item Infine, verranno esaminate le tipologie di minacce specifiche delle architetture \textit{cloud}, preparando il terreno per l'applicazione pratica di questi concetti al modello di Smart Grid proposto nel capitolo successivo.
\end{enumerate}

\section{Esigenze e ambiti applicativi}

% Negli ultimi anni gli attacchi informatici rivolti direttamente ad applicazioni software sono frequenti, causando danni di disponibilità del servizio oltre che a danni economici e finanziari.

% Per limitare che le possibili minacce possano causare danni anche difficilmente riparabili, sempre di più ci si preoccupa di utilizzare delle tecniche di basate sul concetto di \textit{Secure by Design} e l'analisi delle possibili minacce così da prevedere possibili attacchi informatici.

% Di fatto le vulnerabilità all'interno del software sono presenti voi perché introdotte dal team di sviluppo, dalle policy aziendali, dai fornitori, dagli stessi framework o soluzioni open-sorce utilizzate \cite{agid}.

Il panorama della sicurezza informatica è oggi caratterizzato da un'escalation costante di attacchi rivolti alle applicazioni software, con conseguenze che vanno dall'indisponibilità dei servizi a gravi danni economici e reputazionali. Per affrontare questa sfida, l'industria si sta spostando da un approccio puramente reattivo (rispondere agli incidenti dopo che si sono verificati) a un approccio proattivo, basato sul principio del \textit{Secure by Design}.


Questo paradigma impone di integrare la sicurezza in ogni fase del ciclo di vita dello sviluppo del software (SDLC), partendo dal presupposto che le vulnerabilità sono una conseguenza inevitabile della complessità dei sistemi. Le loro origini, infatti, sono molteplici e possono derivare da \cite{agid}: 

\begin{itemize}
    \item Errori di programmazione introdotti dal team di sviluppo.
    \item Debolezze nelle policy di sicurezza aziendali.
    \item Vulnerabilità ereditate da componenti di terze parti, \textit{framework} o librerie \textit{open-source}.
\end{itemize}


In questo contesto, la modellazione delle minacce (\textit{Threat Modeling}) emerge come lo strumento metodologico chiave per implementare il \textit{Secure by Design}. Esso è un processo strutturato che, attraverso l'astrazione e l'analisi del sistema, consente di identificare e ragionare sulle potenziali minacce prima che queste possano essere sfruttate.


L'obiettivo finale del \textit{Threat Modeling} non è solo creare una lista di possibili attacchi, ma fornire le informazioni necessarie per una corretta gestione del rischio. Per ogni minaccia identificata, l'organizzazione può infatti prendere una decisione strategica informata, scegliendo se il rischio debba essere:

\begin{itemize}
    \item \textbf{Mitigato:} applicando una contromisura;
    \item \textbf{Eliminato:} rimuovendo la componente o la funzionalità vulnerabile;
    \item \textbf{Trasferito:} attraverso un'assicurazione o delegando a terzi;
    \item \textbf{Accettato:} se il costo della mitigazione supera il potenziale danno.
\end{itemize}


% L'utilizzo di framework per la ricerca delle minacce che vedremo più avanti permette, attraverso l'astrazione, di aiutare e riflettere su queste vulnerabilità che il sistema, qualsiasi esso sia, possa incorrere e scegliere se la vulnerabilità debba essere: \textbf{Mitigata}, \textbf{Eliminata}, \textbf{Trasferita} o semplicemente \textbf{Accettata}.



\section{Il Processo di Threat Modeling}


"\textit{La modellazione delle minacce è un \textbf{processo strategico}\footnote{Si riferisce alla capacità di anticipare le minacce attraverso modelli di attacco simulati.} volto a prendere in considerazione i possibili scenari di  \textbf{attacchi} e \textbf{vulnerabilità} all'interno di un ambiente applicativo proposto o esistente allo scopo di identificare chiaramente i livelli di \textbf{rischio} e di \textbf{impatto}.}" \cite{libro-threat-modelling}

% \vspace{0.3cm}

L'adozione di questa metodologia offre vantaggi tangibili durante tutto il ciclo di vita dello sviluppo \cite{libro-threat-modelling-designin-for-security}:


% \begin{itemize}
%     \item Consente di individuare possibili bug\footnote{Un bug è un errore nel codice di un programma che causa malfunzionamenti o comportamenti inaspettati, e può rappresentare una vulnerabilità sfruttabile per attacchi informatici.} nelle fasi iniziali dello sviluppo software, cruciale per risparmiare risorse rispetto a trovarlo nelle fasi finali o nel prodotto finito;
%     \item Consente di capire meglio i requisiti di sicurezza, soprattutto considerare i requisiti non presi in considerazione;
%     \item Permette di progettare e consegnare un prodotto migliore, evitando la necessità di riprogettare il sistema;
%     \item Permette di affrontare problemi che altre tecnologie non possono affrontare.
% \end{itemize}


\begin{itemize}
    \item \textbf{Identificazione Precoce dei Difetti:} Permette di individuare vulnerabilità di progettazione nelle fasi iniziali dello sviluppo, riducendo drasticamente i costi di correzione rispetto a un loro rilevamento in fasi successive o dopo il rilascio del prodotto;
    \item \textbf{Definizione dei Requisiti di Sicurezza:} Aiuta a chiarire e a completare i requisiti di sicurezza, evidenziando aspetti inizialmente non considerati;
    \item \textbf{Miglioramento della Progettazione:} Conduce a un'architettura più robusta e sicura, minimizzando la necessità di costose riprogettazioni;
    \item \textbf{Analisi dei Rischi Logici:} A differenza di strumenti automatici che trovano bug\footnote{Un bug è un errore nel codice di un programma che causa malfunzionamenti o comportamenti inaspettati, e può rappresentare una vulnerabilità sfruttabile per attacchi informatici.} nel codice, il \textit{Threat Modeling} è in grado di identificare difetti logici e di progettazione che nessun'altra tecnologia può rilevare;
\end{itemize}

Sebbene il processo possa avvalersi di tecniche collaborative come il \textit{brainstorming}, esso viene tipicamente guidato da uno dei seguenti approcci metodologici  \cite{libro-threat-modelling-designin-for-security}:

% \begin{itemize}
%     \item \textbf{Asset-focused:} consiste nello stilare una lista di asset per poi considerare quali possono essere le minacce che possono subire. 
%     % Possibili attacchi possono essere: 
%     % \begin{itemize}
%     % \item[--] cose che un attaccante vuole come password o secret key
%     % \item[--] cose che si vogliono proteggere come la reputazione di un azienda
%     % \item[--] tutti gli asset già identificati possono essere attaccati e
%     % \item[--] sfruttati per raggiungere altri asset.
%     % \end{itemize}
%     \item \textbf{Attacker-focused:} consiste nello stilare una lista dei possibili attaccanti e cercare di immedesimarsi in loro per capire come il sistema può essere minacciato. Questo approccio è rischioso, perché i programmatori possono proiettare le loro conoscenze nel modellare come si comporterebbe un aggressore, creando un modello che non riflette le possibili minacce reali
%     \item \textbf{Software-focused:} è l'approccio più efficiente. Gli sviluppatori, che hanno la massima conoscenza del software che si sta costruendo, partecipano attivamente per aiutare a creando un modello, di solito attraverso diagrammi, del software e identificando possibili bug.
% \end{itemize}

\begin{enumerate}
    \item \textbf{Approccio Centrato sugli Asset (\textit{Asset-centric}):} Il processo inizia con l'identificazione e la classificazione degli asset critici del sistema (es. dati sensibili, funzionalità chiave). Successivamente, si analizzano le minacce che potrebbero compromettere ciascun asset.
    \item \textbf{Approccio Centrato sull'Attaccante (\textit{Attacker-centric}):} Questo approccio si concentra sulla profilazione dei potenziali attaccanti, analizzandone le motivazioni, le capacità e gli obiettivi. Si cerca quindi di simulare le loro possibili azioni contro il sistema. Sebbene utile, questo metodo presenta il rischio che il team di sviluppo proietti le proprie conoscenze e i propri bias nel modello, sottostimando o ignorando le reali tattiche degli avversari.
    \item \textbf{Approccio Centrato sul Software (\textit{Software-centric}):} Considerato spesso il più efficace in contesti di sviluppo, questo approccio parte da una rappresentazione del sistema stesso, tipicamente attraverso diagrammi di flusso dei dati (DFD). Analizzando come i dati si muovono attraverso i componenti del sistema e superano i confini di fiducia (\textit{trust boundaries}), il team può identificare sistematicamente le potenziali vulnerabilità.
\end{enumerate}

\section{Le Fasi del Processo di Threat Modeling}

% Il processo di modellazione delle minacce risponde a quattro domande essenziali \cite{libro-threat-modelling-designin-for-security,agid}:

% \begin{itemize}
%     \item \textbf{Cosa si sta costruendo?} Capire l'architettura di sistema, componenti, flussi di dati e interazioni.
%     \item \textbf{Cosa può andare storto una volta costruito?} Trovare le potenziali minacce e vulnerabilità che potrà compromettere la sicurezza di sistema, esaminare con cura i possibili attacchi
%     \item \textbf{Che cosa fare per le cose che possono andare storte?} Sviluppo di strategie e contromisure per controllare e/o minimizzare i rischi
%     \item \textbf{Si è fatto un lavoro di analisi decente?} Step di validazione, testi la tua strategia di mitigazione
% \end{itemize}

% Queste domande prevedono un approccio in quattro fasi:

% \begin{center}
%     Modellare il sistema $\rightarrow$ Trovare le minacce $\rightarrow$ Affrontare le minacce $\rightarrow$ Validare il modello
% \end{center}


Il processo di modellazione delle minacce (\textit{Threat Modeling}) è un'attività iterativa che può essere scomposta in quattro fasi fondamentali. Ciascuna fase è progettata per rispondere a una domanda chiave, guidando il team di analisi dalla comprensione del sistema alla validazione delle contromisure implementate \cite{libro-threat-modelling-designin-for-security,agid}.

Di seguito vengono introdotte le quattro fasi, che saranno analizzate in dettaglio nei paragrafi successivi.

\begin{enumerate}
    \item \textbf{Modellazione del Sistema (Risponde a: "Cosa stiamo costruendo?")}\\ La prima fase consiste nel comprendere e rappresentare formalmente il sistema oggetto di analisi. Questo implica la definizione dei suoi componenti, dei confini di fiducia (\textit{trust boundaries}), delle interfacce e, soprattutto, dei flussi di dati (DFD). Un modello accurato è il prerequisito fondamentale per una corretta identificazione delle minacce.
    \item \textbf{Identificazione delle Minacce (Risponde a: "Cosa potrebbe andare storto?")}\\Una volta definito il modello, la seconda fase si concentra sull'identificazione sistematica delle potenziali minacce. Utilizzando framework strutturati come STRIDE, si analizza ogni elemento del modello per enumerare le vulnerabilità che potrebbero comprometterne la sicurezza. L'obiettivo è creare un elenco completo di possibili scenari di attacco.
    \item \textbf{Mitigazione delle Minacce (Risponde a: "Cosa possiamo fare al riguardo?")}\\In questa fase, per ogni minaccia identificata, si definisce una strategia di gestione del rischio. Ciò comporta la progettazione e la prioritizzazione di contromisure di sicurezza (mitigazioni) volte a ridurre la probabilità o l'impatto della minaccia. Le strategie possono includere la modifica della progettazione, l'implementazione di controlli di sicurezza o la revisione delle policy.
    \item \textbf{Validazione delle Mitigazioni (Risponde a: "Abbiamo fatto un buon lavoro?")}\\La fase finale chiude il ciclo verificando che le minacce siano state adeguatamente affrontate. Questo include la revisione delle contromisure implementate, l'esecuzione di test di sicurezza per validarne l'efficacia e l'aggiornamento della documentazione. Questo step garantisce che il processo abbia effettivamente ridotto il livello di rischio del sistema.
\end{enumerate}



\begin{figure}[!h]
    \centering
    \includegraphics[trim= 0cm 0cm 25.5cm 0cm, clip, width=0.15\linewidth]{img/the-4-step-framework-v2.drawio.pdf}
    \caption{Le Fasi del Processo di Threat Modeling}
    \label{fig:4-step-framework}
\end{figure}



\subsection{Modellazione del sistema}

% Il primo passo è quello di capire il sistema nel suo complesso. Questo può essere fatto attraverso una decomposizione, un processo che permette di acquisire conoscenze su come funziona il sistema e come interagisce con le entità esterne. 

% La decomposizione include:

% \begin{itemize}
%     \item la creazione di \textit{casi d'uso - use case}, per identificare le modalità di utilizzo del sistema;
%     \item l'identificazione dei \textit{punti di ingresso - entry points}, che permettono a un aggressore di interagire con il sistema;
%     \item l'identificazione degli \textit{asset} a cui un attaccante potrebbe essere interessato;
%     \item la scoperta di \textit{attori}, utenti che interagiscono con il sistema; essi possono essere interni o esterni e devono ricevere alcuni diritti di accesso;
%     \item l'identificazione dei \textit{livelli di fiducia - trust boundaries}, che determineranno i diritti di accesso per entità esterne.
% \end{itemize}

% Una delle tecniche per decomporre il sistema è la creazione di un Data Flow Diagram (DFD). Questo tipo di diagramma è stato introdotto negli anni '70 per dare una rappresentazione visiva del modo in cui i dati si spostano da un componente all'altro in un sistema o di un'applicazione e dove i dati vengono modificati o memorizzati (temporaneamente o a lungo termine) all'interno del sistema.

% La DFD identifica chiaramente le \textit{Entità esterne}, i \textit{Punti finali} del sistema, i \textit{processi},e le \textit{unità di funzione}, il \textit{Data Flow} (DF) e il \textit{Data Store} (DS).
% Più tardi, nei primi anni 2000, è stato aggiunto il concetto di \textit{trust boundaries}\footnote{ Un trust boundaries definisce il punto dove cambia il livello di fiducia tra due componenti o domini di un sistema} per migliorare le DFD.
% I trust boundaries vengono utilizzati per isolare gli elementi attendibili e non attendibili. \cite{STRIDE-paper}


La prima e fondamentale fase del processo di \textit{Threat Modeling} consiste nel creare una rappresentazione astratta ma accurata del sistema da analizzare. L'obiettivo è comprendere a fondo i suoi componenti, le interazioni e, soprattutto, come i dati fluiscono e vengono trattati al suo interno.


Lo strumento standard per questa attività è il \textit{Data Flow Diagram} (DFD). Introdotto originariamente negli anni '70, il DFD è una tecnica di rappresentazione grafica che visualizza il flusso di informazioni all'interno di un sistema. Invece di mostrare la logica di controllo (come farebbe un \textit{flowchart}), un DFD si concentra esclusivamente sul movimento e sulla trasformazione dei dati.


Un DFD è composto da quattro elementi fondamentali:


\begin{enumerate}
    \item \textbf{Entità Esterne (\textit{External Entities}):} Rappresentano gli attori, sia umani che altri sistemi, che interagiscono con il sistema inviando o ricevendo dati, ma che si trovano al di fuori del suo controllo (es. un utente, un'API di terze parti).
    \item \textbf{Processi (\textit{Processes}):} Sono le componenti del sistema che elaborano o trasformano i dati. Ogni processo prende dei dati in input e produce dei dati in output.
    \item \textbf{Archivio dati (\textit{Data Store}):} Rappresentano i luoghi in cui i dati vengono archiviati, sia in modo temporaneo (es. una cache) che permanente (es. un database).
    \item \textbf{Flussi di Dati (Data Flows):} Sono le frecce che collegano gli altri elementi del diagramma, indicando la direzione in cui i dati si muovono.
\end{enumerate}

Per l'analisi di sicurezza, i DFD sono stati arricchiti con un quinto elemento cruciale: i Confini di Fiducia (\textit{Trust Boundaries}). Questi confini sono linee tratteggiate che delimitano le aree del sistema con diversi livelli di privilegio o fiducia. Un flusso di dati che attraversa un trust boundary rappresenta un punto di ingresso critico (\textit{entry point}) e una potenziale superficie di attacco che richiede un'analisi particolarmente attenta \cite{STRIDE-paper}.


La costruzione di un DFD costringe il team a rispondere a domande essenziali: quali sono gli asset da proteggere? Chi sono gli attori che interagiscono con il sistema? Quali sono i punti di ingresso e come vengono validati i dati che li attraversano? Questo modello diventa così la mappa su cui, nella fase successiva, verranno sistematicamente identificate le minacce.



\subsection{Identificazione  delle minacce}

% Questo è l'elemento centrale della modellazione delle minacce: le minacce e gli agenti di minaccia\footnote{Agente di minaccia: individuo o gruppo interessato a sfruttare una vulnerabilità e a realizzare un'azione di
% minaccia contro l'asset} possono essere essere identificati seguendo diversi diverse metodologie: 

% \begin{itemize}
%     \item STRIDE
%     \item Attack threes
%     \item PASTA
%     \item CVSS
%     \item Security Cards.
% \end{itemize}

% Per lo scopo di questa tesi verrà utilizzata la prima metodologia: STRIDE.

Una volta ottenuto un modello chiaro del sistema attraverso il DFD, la seconda fase del processo consiste nell'identificare sistematicamente le minacce. Questa attività, spesso definita "\textit{threat enumeration}", ha lo scopo di rispondere alla domanda: "Cosa potrebbe andare storto?". Si analizza ogni componente del DFD (processi, flussi di dati, data store) per individuare le potenziali vulnerabilità.


Per guidare questa analisi in modo strutturato e ripetibile, sono stati sviluppati numerosi framework e metodologie. Tra i più noti si includono:


\begin{itemize}
    \item \textbf{STRIDE:} Un modello di classificazione delle minacce sviluppato da Microsoft, focalizzato sulle proprietà di sicurezza che un software dovrebbe garantire.
    \item \textbf{Attack Trees: }Una tecnica che scompone un potenziale attacco in una struttura ad albero, mappando i passaggi necessari per raggiungere un obiettivo malevolo.
    \item \textbf{PASTA (\textit{Process for Attack Simulation and Threat Analysis}):} Una metodologia completa in sette fasi che allinea le minacce agli obiettivi di business.
    \item \textbf{CVSS (\textit{Common Vulnerability Scoring Syste}m):} Sebbene non sia una metodologia di \textit{threat modeling}, è un sistema di punteggio utilizzato per valutare la gravità delle vulnerabilità una volta identificate.
    \item \textbf{LINDDUN:} Un framework specifico per l'identificazione di minacce alla privacy.
\end{itemize}


Per l'analisi condotta in questa tesi, è stato scelto il framework STRIDE. Questa decisione è motivata dalla sua stretta integrazione con la modellazione basata su DFD e dalla sua efficacia nell'identificare un'ampia gamma di minacce a livello di progettazione software. La sua natura sistematica lo rende particolarmente adatto ad analizzare sistemi complessi e distribuiti come l'architettura Smart Grid \textit{Cloud-Native} proposta. Il framework STRIDE verrà descritto in dettaglio nella sezione seguente.


\subsection{Mitigazione delle Minacce}

% Una volta identificate le minacce, è necessario comprendere come affrontarle, quali sono più urgenti da gestire e quali contromisure di sicurezza sono necessarie per mitigare il loro impatto. Una tecnica utile è creare una matrice di tracciabilità delle minacce: gli attacchi vengono elencati sulla base del pericolo ad essi associato. Tenendo conto del livello di rischio associato a ciascuna vulnerabilità durante il processo di gestione del rischio, le minacce vengono classificate da più gravi a meno gravi, e le parti interessate e i proprietari del rischio possono analizzarle per trovare le contromisure appropriate e le tecniche di mitigazione. I rischi verranno trattati secondo quanto definito nel processo di gestione del rischio.

Una volta completata l'identificazione delle minacce, la terza fase del processo si concentra su come affrontarle. Non è sufficiente avere una lista di potenziali attacchi; è necessario valutarli, prioritizzarli e definire contromisure adeguate. Questo processo si articola in tre attività principali.

% \vspace{-0.11cm}
\begin{enumerate}
    \item \textbf{Valutazione del Rischio (Risk Assessment):}
    Per ogni minaccia identificata, viene effettuata una valutazione del rischio associato. Questo non si basa solo sulla natura della minaccia stessa, ma su una combinazione di due fattori chiave:
    % \vspace{-0.11cm}
    \begin{itemize}
        \item \textbf{Probabilità (\textit{Likelihood}):} La probabilità che la vulnerabilità possa essere effettivamente sfruttata da un attaccante.
        \item \textbf{Impatto (\textit{Impact}):} Il danno potenziale (operativo, finanziario, reputazionale) che si verificherebbe in caso di successo dell'attacco.
    \end{itemize}
    % \vspace{-0.11cm}
     Molte metodologie, come DREAD, assegnano un punteggio a questi fattori per calcolare un livello di rischio complessivo per ogni minaccia.
     % \vspace{-0.11cm}
    \item \textbf{Prioritizzazione delle Minacce:}
    Sulla base del livello di rischio calcolato, le minacce vengono classificate in ordine di priorità, da quelle più critiche a quelle meno gravi. Questo permette al team di concentrare le risorse e l'attenzione sulla risoluzione dei problemi che rappresentano il maggior pericolo per il sistema.
    % \vspace{-0.11cm}
    \item \textbf{Definizione delle Contromisure:}
    Per le minacce prioritarie, si passa alla progettazione delle contromisure (o mitigazioni). L'obiettivo è applicare controlli di sicurezza che riducano la probabilità o l'impatto della minaccia a un livello accettabile. Come già discusso nel contesto della gestione del rischio, le opzioni non si limitano alla mitigazione; il team può decidere di eliminare una funzionalità, trasferire il rischio o accettarlo consapevolmente.
\end{enumerate}

% \vspace{-0.11cm}

% Per documentare questo processo, si utilizza spesso una matrice di tracciabilità delle minacce, che associa a ogni minaccia identificata il suo livello di rischio, la contromisura proposta e lo stato di implementazione, garantendo che nessuna vulnerabilità nota venga trascurata.


\subsection{Validazione delle Mitigazioni}


% Quest'ultimo passaggio consiste nel verificare se il modello di minaccia è completo (se identifica tutte le possibili minacce) e se tutte le minacce sono adeguatamente mitigate. Per le minacce che non sono state completamente mitigate, il rischio residuo viene calcolato e analizzato (il rischio residuo è coerente con il livello di rischio accettabile?).[13] Una strategia comune per validare il modello di minaccia è l'utilizzo di test. I test possono essere automatici o manuali e possono essere applicate diverse tecniche. Ad esempio, il penetration testing può essere utilizzato per valutare le vulnerabilità del sistema (le vulnerabilità trovate vengono confrontate con quelle identificate dal modello di minaccia), oppure le tecniche di mitigazione identificate dal modello vengono validate simulando un attacco e applicando quelle tecniche per verificare se sono realmente in grado di mitigare l'attacco.

La fase finale del processo di Threat Modeling chiude il ciclo, assicurando che il lavoro svolto abbia effettivamente migliorato la postura di sicurezza del sistema. Questa fase di validazione ha un duplice obiettivo: verificare la completezza dell'analisi e l'efficacia delle contromisure implementate. Le attività principali includono:

% \vspace{-0.11cm}
\begin{enumerate}
    \item \textbf{Revisione del Modello e delle Contromisure:}
    Si riesamina l'intero modello di minaccia per confermarne l'accuratezza e la completezza. Il team si assicura che tutte le minacce identificate siano state associate a una strategia di gestione del rischio e che le contromisure progettate siano state implementate correttamente secondo le specifiche.
    % \vspace{-0.11cm}
    \item \textbf{Analisi del Rischio Residuo:}
    È raro che tutte le minacce possano essere eliminate completamente. Per le minacce che sono state mitigate (ma non eliminate) o accettate, si valuta il rischio residuo, ovvero il livello di rischio che permane nel sistema dopo l'applicazione dei controlli di sicurezza. È compito dei responsabili del rischio (\textit{risk owner}) determinare se tale rischio residuo rientri nella soglia di tolleranza definita dall'organizzazione.
    % \vspace{-0.11cm}
    \item \textbf{Test di Sicurezza e Validazione Pratica:}
    Per verificare empiricamente l'efficacia delle mitigazioni, si ricorre a test di sicurezza. Questi possono includere:
    % \vspace{-0.11cm}
    \begin{itemize}
        \item \textbf{\textit{Penetration Testing}:} Viene eseguito un attacco simulato da parte di "\textit{ethical hacker}" per tentare di sfruttare le vulnerabilità del sistema. I risultati vengono poi confrontati con le minacce identificate nel modello per verificarne la copertura.
        \item \textbf{\textit{Security Test Case}:} Vengono creati casi di test specifici per validare che ogni singola contromisura funzioni come previsto (es. "verificare che un input malizioso venga correttamente rigettato dal sistema di validazione").
    \end{itemize}
\end{enumerate}
% \vspace{-0.11cm}

Solo al termine di questa fase di validazione si può considerare concluso un ciclo di \textit{Threat Modeling}. Il modello, tuttavia, non è un documento statico: deve essere rivisto e aggiornato ogni volta che il sistema subisce modifiche significative.

\section{Il framework utilizzato: STRIDE}

% È stato creato da Microsoft per tre scopi principali \cite{STRIDE-paper}: 

% \begin{enumerate}
%     \item come approccio sistematico per analizzare le possibili minacce informatiche contro ogni componente del sistema basandosi sulla sua conoscenza tecnica;
%     \item per fornire un'analisi completa delle proprietà di sicurezza;
%     \item per identificare l'impatto della vulnerabilità di un componente sull'intero sistema.
% \end{enumerate}

% STRIDE è l'acronimo per il tipo di minacce che copre \cite{libro-threat-modelling-designin-for-security}:

% \begin{itemize}
%     \item \textbf{Spoofing:} violazione della proprietà di autenticazione. L'attaccante finge di essere qualcun altro, come un processo, un'entità esterna o una persona e compromette qualcosa all'interno del sistema. Uno scenario comune può essere: l'attaccante sfrutta un sistema di autenticazione debole (ad esempio, intercettando la chiave da API che utilizzano richieste di autenticazione a chiave singola). Una volta rubata la chiave e ottenuto l'accesso al sistema, l'attaccante finge di essere un processo innocuo e crea o modifica maliziosamente un file prima del processo reale.
%     \item \textbf{Tampering:} violazione della proprietà di integrità. L'attaccante modifica qualcosa (un file, il codice o alcuni dati) in modo non autorizzato. Questo attacco potrebbe essere rilevato controllando i file di log e le notifiche.
%     \item \textbf{Repudiation:} violazione della proprietà di non ripudio. Garantisce che un comportamento scorretto non possa essere provato. Alcuni meccanismi di non ripudio potrebbero essere l'auditing e il tracciamento (ma considerando sempre che anche i file di tracciamento potrebbero essere manomessi).
%     \item \textbf{Information disclosure:} violazione della proprietà di riservatezza. Alcune informazioni riservate potrebbero essere accidentalmente divulgate (ad esempio, attraverso messaggi di errore) o essere esposte a un attacco (come il buffer overflow).
%     \item \textbf{Denial of Service (DoS):} violazione della proprietà di disponibilità. Un sistema diventa irraggiungibile sfruttando maliziosamente le sue risorse e impedendogli di essere utilizzato per scopi legittimi. Alcuni esempi possono essere DoS che colpiscono i processi (l'attacco assorbe memoria o CPU e il processo non è in grado di funzionare) o database (vengono riempiti con informazioni inutili e non sono in grado di ricevere dati utili).
%     \item \textbf{Elevation of privilege:} violazione della proprietà di autorizzazione. L'attaccante dichiara di essere un utente autorizzato con privilegi elevati (come admin invece di un utente comune). Ad esempio, corrompendo un processo, l'attaccante può ottenere diritti di accesso in lettura o scrittura a alcune posizioni di memoria sensibili.
% \end{itemize}


% Una volta creato il DFD, la modellazione delle minacce basata su STRIDE può essere eseguita in due modi: 

% \begin{itemize}
%     \item STRIDE per-elemento: per ogni minaccia coperta da STRIDE, ogni componente del sistema viene analizzato per verificare se può essere soggetto a questa minaccia.
%     \item STRIDE-per-interazione: i componenti del sistema sono considerati in tuple (origine, destinazione e interazione) e la loro interazione viene analizzata per verificare se può essere soggetta a una o più minacce coperte da STRIDE.
% \end{itemize}

Sviluppato originariamente da Microsoft, STRIDE è un modello di classificazione delle minacce che aiuta gli analisti a identificare sistematicamente un'ampia gamma di vulnerabilità di sicurezza. Il suo scopo è fornire un approccio mnemonico e strutturato per ragionare sulle possibili minacce contro ogni componente di un sistema, mappando ogni minaccia a una specifica proprietà di sicurezza che viene violata \cite{STRIDE-paper, libro-threat-modelling-designin-for-security}.

L'acronimo STRIDE rappresenta sei categorie di minacce:
\begin{itemize}
    \item \textit{\textbf{S}poofin}g (Falsificazione dell'identità): Si verifica quando un aggressore si finge illegittimamente un altro utente, componente o sistema. Viola la proprietà di Autenticazione.
    % * Esempio: Un attaccante ruba le credenziali di un utente e accede al sistema impersonandolo, oppure un processo malevolo si finge un processo legittimo per ricevere dati sensibili.
    \item \textit{\textbf{T}ampering} (Manomissione): Consiste nella modifica non autorizzata di dati, sia in transito su una rete che archiviati in un data store. Viola la proprietà di Integrità.
    % * Esempio: Un attaccante intercetta un flusso di dati e ne altera il contenuto, oppure modifica direttamente un file di configurazione o un record in un database.
    \item \textit{\textbf{R}epudiation} (Ripudio): Si riferisce alla capacità di un utente di negare di aver compiuto un'azione, in assenza di prove che dimostrino il contrario. Viola la proprietà di Non Ripudio.
     % * Esempio: Un utente esegue un'operazione dannosa e poi cancella i file di log per eliminare le tracce, rendendo impossibile attribuirgli l'azione.
    \item \textit{\textbf{I}nformation Disclosure} (Rivelazione di informazioni): Consiste nell'esposizione di informazioni sensibili a soggetti non autorizzati. Viola la proprietà di Confidenzialità.
    % * Esempio: Un messaggio di errore che rivela dettagli interni del sistema, un accesso non autorizzato a un database, o una vulnerabilità come un buffer overflow che espone dati in memoria.
    \item \textit{\textbf{D}enial of Service} (DoS - Negazione del servizio): Si verifica quando un attaccante rende un sistema o una risorsa non disponibile per gli utenti legittimi. Viola la proprietà di Disponibilità.
    % * Esempio: Un attacco che esaurisce la CPU o la memoria di un processo, o che inonda una rete di traffico inutile per renderla inaccessibile.
    \item \textit{\textbf{E}levation of Privilege} (EoP - Acquisizione di privilegi): Avviene quando un utente con privilegi limitati riesce a ottenere accessi o permessi superiori a quelli che gli sono stati assegnati. Viola la proprietà di Autorizzazione.
    % * Esempio: Un utente standard sfrutta una vulnerabilità per ottenere i privilegi di amministratore, ottenendo così accesso a funzionalità e dati riservati.
\end{itemize}

La forza di STRIDE risiede nella sua applicazione sistematica a un \textit{Data Flow Diagram}.
Una volta creato il DFD, la modellazione delle minacce basata su STRIDE può essere eseguita in due modi: 

\begin{itemize}
    \item STRIDE per-elemento: per ogni minaccia coperta da STRIDE, ogni componente del sistema viene analizzato per verificare se può essere soggetto a questa minaccia.
    \item STRIDE-per-interazione: i componenti del sistema sono considerati in tuple (origine, destinazione e interazione) e la loro interazione viene analizzata per verificare se può essere soggetta a una o più minacce coperte da STRIDE.
\end{itemize}

% \section{Minacce informatiche su architettura cloud}

% \newpage
% \chapter{Applicazione del Threat Modeling all'Architettura Proposta}

% % \section{Applicazione del Threat Modeling: Modellazione del Sistema}


% % Giunti a questo punto, dopo aver analizzato i passi da seguire per modellizzare le minacce possiamo iniziare con il definire il Data Flow Diagram del nostro sistema Smart Grid.

% % \begin{figure}[!h]
% %     \centering
% %     \includegraphics[trim= 0cm 39cm 0cm 0cm, clip, width=0.8\linewidth]{img/DFD.drawio.pdf}
% %     \caption{Data flow diagram - Smart Grid Cloud-Native}
% %     \label{fig:DFD}
% % \end{figure}

% % Come si vede dalla Figura \ref{fig:DFD}, visione semplificata delle componenti cardine, sono stati identificati dei \textit{Boundaries}, ovvero dei confini sicuri, che possono essere di due tipologie:

% % \begin{itemize}

% %     \item Trust Boundaries (rossi): generici confini di sicurezza, garantiti di fatto dalla presenta dell'architettura Cloud-Native.
% %     \item Machine Boundaries (blu): questi sono dei confini di sicurezza fisici, che proteggono i componenti sul campo da possibili manomissioni.
% % \end{itemize}

% Nei capitoli precedenti sono stati definiti i tre pilastri concettuali di questa tesi: il concetto di Smart Grid con la sua architettura e componenti (Capitolo 1), un'implementazione di una Smart Grid basata su un paradigma Cloud-Native (Capitolo 2) e la metodologia formale del Threat Modeling per l'analisi della sicurezza dei sistemi (Capitolo 3).

% Questo capitolo rappresenta il punto di convergenza di questi tre elementi, costituendo il contributo centrale della ricerca. L'obiettivo è applicare sistematicamente il processo di Threat Modeling, e in particolare il framework STRIDE, al modello architetturale proposto, al fine di identificare e classificare le principali minacce informatiche che lo caratterizzano.


% L'analisi seguirà fedelmente le quattro fasi metodologiche descritte in precedenza:

% \begin{enumerate}
%     \item \textbf{Modellazione del Sistema:} Verrà presentato e discusso in dettaglio il Data Flow Diagram (DFD) dell'architettura.
%     \item \textbf{Identificazione delle Minacce:} Ogni componente del DFD verrà analizzato attraverso la lente di STRIDE per enumerare le potenziali minacce.
%     \item \textbf{Mitigazione delle Minacce:} Per le minacce più significative, verranno proposte delle contromisure di sicurezza specifiche per il contesto Cloud-Native.
%     \item \textbf{Validazione:} Verranno discusse le strategie per la validazione delle mitigazioni proposte.
% \end{enumerate}


% % \section{Modellazione del Sistema}
% \section{Definizione del Data Flow Diagram dell'Architettura}

% In questa sezione si avvia l'applicazione pratica del processo di Threat Modeling all'architettura Smart Grid Cloud-Native proposta. Il primo passo fondamentale, come descritto dalla metodologia, consiste nella modellazione del sistema attraverso un Data Flow Diagram (DFD).

% La  Figura \ref{fig:DFD} presenta un DFD di Livello 0 che astrae l'architettura, evidenziandone i componenti principali, i flussi di dati e, soprattutto, i confini di sicurezza. In questo modello sono stati identificati i seguenti elementi:

% \begin{itemize}
%     \item \textbf{Entità Esterne:} Componenti che interagiscono con il sistema ma si trovano al di fuori del suo controllo diretto, come lo Smart Meter, l'RTU/IED e il sistema PMU/Phasor Data Concentrator.
%     \item \textbf{Processi:} I componenti software che elaborano i dati, come il Data Concentrator, la Rete AMI (che agisce come hub di comunicazione), il DMS e l'EMS.
%     \item \textbf{Archivi Dati:} I luoghi di memorizzazione dei dati, rappresentati dai Cloud Store.
% \end{itemize}

% Per l'analisi di sicurezza, sono stati definiti due tipi di confini (boundaries), ciascuno con un significato preciso:

% \begin{enumerate}
%     \item \textbf{Trust Boundary (confine rosso):} Rappresenta un confine logico che separa componenti con diversi livelli di fiducia. Qualsiasi flusso di dati che attraversa un Trust Boundary (ad esempio, dalla "Rete AMI" al processo "DMS") deve essere considerato potenzialmente ostile e quindi soggetto a rigorose procedure di autenticazione, autorizzazione e validazione. Questi confini definiscono la superficie di attacco di ciascun servizio.
%     \item \textbf{Machine Boundary (confine blu):} Rappresenta un confine fisico o a livello di dispositivo. Esso isola i componenti che operano sul campo (Edge), come il Data Concentrator o il Phasor Data Concentrator, dal loro ambiente fisico e dalla rete locale. Questo confine è rilevante per analizzare minacce di accesso fisico (manomissione) o attacchi diretti al dispositivo, che bypasserebbero i controlli a livello di applicazione.
% \end{enumerate}

% % Questo DFD, con la sua chiara distinzione tra processi, dati e confini, costituisce la mappa fondamentale su cui, nella fase successiva, verranno sistematicamente identificate le minacce utilizzando il framework STRIDE.



% \begin{figure}[!h]
%     \centering
%     \includegraphics[trim= 0cm 39cm 0cm 0cm, clip, width=0.8\linewidth]{img/DFD.drawio.pdf}
%     \caption{Data flow diagram - Smart Grid Cloud-Native}
%     \label{fig:DFD}
% \end{figure}


% \subsection{Definizione dell'Ambito di Analisi}

% % Nella Tabella \ref{tab:def-ambito} troviamo la definizione degli elementi: \textit{In scope}, ovvero all'interno dell'ambito di ricerca di questa tesi; \textit{Out od scope}, che non fanno parte prettamente all'infrastruttura cloud della Smart Grid. 

% Prima di procedere con l'analisi delle minacce, è essenziale definire con precisione il perimetro (o scope) di questo studio. Data la vastità dell'ecosistema Smart Grid, l'analisi si concentrerà specificamente sulle vulnerabilità introdotte dalla sua implementazione in un'architettura Cloud-Native.


% Di conseguenza, verranno prese in considerazione le minacce relative ai componenti software centralizzati e ai canali di comunicazione che li collegano al campo. 
% % Le problematiche di sicurezza legate strettamente all'hardware dei dispositivi Edge – come la manomissione fisica, gli attacchi a livello di firmware o le vulnerabilità dei circuiti integrati – pur essendo di fondamentale importanza per la sicurezza complessiva, sono considerate al di fuori dell'ambito di questa tesi per mantenere un focus mirato sulle minacce a livello di architettura di rete e applicativa.


% La Tabella \ref{tab:def-ambito} riassume formalmente questa suddivisione, elencando gli elementi considerati oggetto di analisi (\textit{In Scope}) e esclusi dall'analisi (\textit{Out of Scope}).

% % \begin{table}[h!]
% %     \centering
% %     % \renewcommand{\arraystretch}{1.5}
    
% %     \begin{tabular}{c|c|c}
% %          &  \textbf{In scope} & \textbf{Out of scope}\\
% %          \hline
% %          &  & \\
% %          Componenti  &   HES, MDMS, DMS, EMS, & Sicurezza dei dispositivi hardware: SM, DC, \\         
% %          software: &   SCADA, GMS, High-level PDC & RTU/IED, Generatori, low-level PDC, PMU\\
% %         &  & \\
         
% %          Canali di &  PLC/RF $169\,MHz$, 4G/5G,&  Sicurezza della cabina Telco\\
% %          comunicaizone: &    Fibra, VPN  & \\
% %           &  & \\
% %          Infrastruttura cloud:&  Kubernetes e container & \\
% %           &  & \\
% %     \end{tabular}
% %     \caption{Definizione dell'ambito}
% %     \label{tab:def-ambito}
% % \end{table}

% \renewcommand{\arraystretch}{1.5}
% \begin{longtable}[!h]{p{5cm}p{5cm}p{5cm}}
        
%     \caption{Definizione dell'ambito di analisi} 
%     \label{tab:def-ambito}\\
    
%     \hline
%     &  \textbf{Oggetto in analisi} & \textbf{Oggetto fuori dall'analisi}\\
%     \hline
%     \endfirsthead
    
%     \hline
%     &  \textbf{Oggetto in analisi} & \textbf{Oggetto fuori dall'analisi}\\
%     \hline
%     \endhead

%     Componenti software: &   SM, HES, MDMS, DMS, EMS, SCADA, GMS, High-level PDC & Sicurezza dei dispositivi hardware: DC, RTU/IED, Generatori, low-level PDC, PMU\\
    
%     Canali di comunicaizone: &  PLC/RF $169\,MHz$, 4G/5G, Fibra, VPN&  Sicurezza della cabina Telco\\
         
%     Infrastruttura cloud:&  Kubernetes e container & \\

    
%     \hline
% \end{longtable}


% \subsection{Identificazione degli Asset Critici}

% % Nella seguente Tabella \ref{tab:def-asset}, in ogni colonna, sono presentati una serie di asset


% % \begin{table}[h!]
% %     \centering
% %     \renewcommand{\arraystretch}{1.5}
    
% %     \begin{tabular}{c|c|c}
% %          \textbf{Dati} &  \textbf{Processi} & \textbf{Infrastruttura logica}\\
% %          \hline
% %          Statistiche del cliente   &    dati collezionati da HES & Cluster K8s  \\         
% %          Fatturazione consumi (MDMS) &   elaborazione dei dati MDMS & Immagini container \\
% %          Statistiche di rete (PMU) &  monitoraggio della rete EMS/DMS & Infrastruttura VPN \\
         
% %          API, credenziali, token & Dati collezionati da high-level PDC & Cloud storage  \\

% %     \end{tabular}
% %     \caption{Definizione degli asset}
% %     \label{tab:def-asset}
% % \end{table}



% Successivamente alla definizione della struttura del sistema tramite il DFD, è cruciale identificare gli asset, ovvero gli elementi di valore all'interno dell'architettura la cui compromissione causerebbe un danno significativo. Sapere cosa si sta proteggendo è un prerequisito fondamentale per poter valutare l'impatto reale di una minaccia.
% Un'analisi completa considera che gli asset non sono limitati ai soli dati, ma includono anche i processi e i componenti infrastrutturali che li gestiscono. Per questa tesi, gli asset sono stati classificati in tre tipologie principali:

% \begin{enumerate}
%     \item \textbf{Dati:} Rappresentano le informazioni sensibili o critiche gestite dal sistema. La loro compromissione può portare a violazioni della privacy, frodi o perdita di controllo sulla rete. Esempi includono i dati di consumo dei clienti, le statistiche di rete delle PMU e le credenziali di accesso come token e chiavi API.
%     \item \textbf{Processi:} Sono le funzioni operative e di elaborazione chiave del sistema. Un attacco a un processo può corrompere i dati, causare un'interruzione del servizio o portare a decisioni errate nella gestione della rete. Esempi includono l'elaborazione dei dati da parte dell'MDMS o il monitoraggio della rete da parte dell'EMS.
%     \item \textbf{Infrastruttura Logica:} Costituisce la base tecnologica su cui poggia l'intera architettura. La compromissione di questi componenti può avere un impatto a cascata su tutti i servizi ospitati. Esempi includono i cluster Kubernetes, le immagini dei container e l'infrastruttura VPN che garantisce la comunicazione sicura.
% \end{enumerate}

% La Tabella \ref{tab:def-asset} presenta una sintesi dei principali asset identificati per l'architettura in esame, classificati secondo queste tre tipologie.

% \renewcommand{\arraystretch}{1.5}
% \begin{longtable}[!h]{p{5cm}p{5cm}p{5cm}}
        
%     \caption{Identificazione degli Asset Critici}
%     \label{tab:def-asset}\\
    
%     \hline
%     \textbf{Dati} &  \textbf{Processi} & \textbf{Infrastruttura logica}\\
%     \hline
%     \endfirsthead
    
%     \hline
%     \textbf{Dati} &  \textbf{Processi} & \textbf{Infrastruttura logica}\\
%     \hline
%     \endhead
    
    
%     Statistiche del cliente   &    dati collezionati da HES & Cluster K8s  \\         
%     Fatturazione consumi (MDMS) &   elaborazione dei dati MDMS & Immagini container \\
%     Statistiche di rete (PMU) &  monitoraggio della rete EMS/DMS & Infrastruttura VPN \\    
%     API, credenziali, token & Dati collezionati da high-level PDC & Cloud storage  \\
    
    
%     \hline
% \end{longtable}


% \section{Analisi delle Minacce con il Framework STRIDE}

% Dopo aver modellato il sistema, si procede ora con la fase di identificazione delle minacce, il cuore di questa analisi. Come anticipato, questa fase verrà condotta applicando sistematicamente il framework STRIDE e utilizzando come riferimento il Data Flow Diagram Figura \ref{fig:DFD}.


% L'approccio utilizzato sarà quello di \textbf{STRIDE-per-elemento}: per ogni componente del DFD (processi, data store, flussi di dati ed entità esterne), verranno considerate le categorie di minaccia STRIDE pertinenti. Questo metodo garantisce una copertura completa e strutturata, riducendo il rischio di tralasciare vulnerabilità significative.




% % \renewcommand{\arraystretch}{1.5}
% % \begin{longtable}{p{1.5cm}p{2cm}p{2cm}p{6.5cm}p{3cm}}
% %     \caption{Minacce informatiche}
% %     \label{tab:minacce-info} \\
    
% %     \hline
% %     \textbf{ID Minaccia} &\textbf{Elemento} & \textbf{Categoria STRIDE}& \textbf{Descrizione minaccia} & \textbf{Possibile attaccante} \\
% %     \hline
% %     \endfirsthead
    
% %     \hline
% %     \textbf{ID Minaccia} &\textbf{Elemento} & \textbf{Categoria STRIDE} & \textbf{Descrizione minaccia} &\textbf{Possibile attaccante} \\
% %     \hline
% %     \endhead

% %     S-01 & Cluster K8s & \textbf{S}poofing, Elevation of Privilege & Service Account Impersonation: un attaccante sfrutta una vulnerabilità di un pod per impossessarsi della private key del suo Service Account e lo usa per accedere alle API di K8s, impersonando il servizio per leggere configurazione e lanciare altri pod malevoli. & Attaccante esterno che ha ottenuto un accesso iniziale; insider\\
    
% %     T-01 & Nodi worker K8s &  \textbf{T}ampering, Elevation of Privilege & Un attaccante sfruttando una vulnerabilità del cloud provider, modifica in una qualsiasi parte il worker node di K8s permettendogli di inserire del codice malevole per prenderne il possesso. & Attaccante molto competente\\

% %     T-02 & Comunicazioni & \textbf{T}ampering & Un attaccante riesce ad intercettare la comunicazione, da e/o verso un servizio cloud, modificandone i dati di consumo o comandi di esecuzione. Man-in-the-middle attack. & Attaccante esterno con accesso alla rete di comunicazione (PLC, VPN, Fibra)\\

% %     R-01 & Invio comandi & \textbf{R}epudiation & Un operatore di un sistema (EMS/DMS) invia un comando alla rete, di trasformazione, produzione, ecc., legittimo ma dannoso per errore e poi nega di averlo fatto. La mancanza di log immutabili e firmati digitalmente rende impossibile l'attribuzione del fatto. & Operatore interno \\
    

% %     I-01 & Flussi dati & \textbf{I}nformation Disclosure & A causa di regole di firewall assenti o troppo permissive tra i cluster K8s, un attaccante può sfruttare le porte lasciate aperte per sferrare l'attacco & Attaccante esterno \\

% %     I-02 & Comunicazione in entrata & \textbf{I}nformation Disclosure & Le regole impostate del firewall risultano essere troppo permissive, esponendo su internet i nodi dei cluster, incluse porte sensibili come la porta 22 (SSH) la quale può essere presa di mira dagli attaccanti. & Attaccante esterno \\

% %     I-03 & Dati salvati & \textbf{I}nformation Disclosure & Un utente interno o un possibile attaccante sfruttando configurazioni errate del cloud storage per accedere ai dati dei clienti o allo stato della rete. & Insider; Attaccante esterno\\

% %     D-01 & Disponibilità servizio & \textbf{D}enial of Service & Un attaccante lancia un attacco DDoS contro gli endpoint della della rete, impedendo agli operatori l'analisi della rete in tempo reale & Attaccante esterno \\

% %     E-01 & Cluster K8s & \textbf{E}levation of Privilege &  Non rispetto del principio del minimo privilegio, assegnando al personale ruoli di accesso non pertinenti alla sua mansione & Insider\\

% %     E-02 & \textbf{E}levation of Privilege & Processi interni al cluster K8s & Un attaccante, dopo aver compremesso un pod con bassi privilegi, sfrutta una vulnerabilità del container runtime o del kernel per ottenere l'accesso root sul nodo, potendo così inviare messaggi malevoli alla rete & Attaccante esterno \\
   

% % \hline

% % \end{longtable}


% \renewcommand{\arraystretch}{1.5}
% \begin{longtable}{p{1.5cm}p{2cm}p{2cm}p{7.5cm}p{2cm}}
%     \caption{Minacce informatiche}
%     \label{tab:minacce-info} \\
    
%     \hline
%     \textbf{ID Minaccia} &\textbf{Elemento} & \textbf{Categoria STRIDE}& \textbf{Descrizione minaccia} & \textbf{Possibile attaccante} \\
%     \hline
%     \endfirsthead
    
%     \hline
%     \textbf{ID Minaccia} &\textbf{Elemento} & \textbf{Categoria STRIDE} & \textbf{Descrizione minaccia} &\textbf{Possibile attaccante} \\
%     \hline
%     \endhead

%     S-01 & Smart Meter & \textbf{S}poofing & Una mancanza di risorse e memoria limitata su SM e dispositivi di campo, possono impedire l'implementazione di tutte le funzionalità di sicurezza e l'aggiornamento del firmware, rendendoli più vulnerabili ad attaccanti esperti che riescono a sfruttare queste vulnerabilità minando l'affidabilità dei dati inviati ai DC. \cite{paper-threat-modelling}& Attaccante esterno\\
    
%     T-01 & Flusso dati PDC & \textbf{T}ampering & Un attaccante non si limita ad un possibile DoS bloccando i dati inviati dai PDC su rete 4G/5G, bensì cerca di intercettare il traffico e modificare leggermente e costantemente tutti i dati prima che essi raggiungano l'EMS. Questo attacco simula un \textbf{carico fantasma} o una falsa instabilità di frequenza. L'EMS che si fida di questi dati reagisce automaticamente ridirigendo l'energia o sezionando la zona, causando gravi disagi. \cite{paper-threat-modelling}& Attaccante esterno\\

%     R-01 & Dati consumatori & \textbf{R}epudiation & Un attaccante esterno, riuscendo ad avere accesso privilegiato con permessi di lettura e scrittura al cloud store dell'infrastruttura AMI, riesce a modificare i dati contenuti nel database con conseguente impatto sulle misure fino ad ora effettuate e possibili previsioni future. & Attaccante esterno \\
    

%     I-01 & Software di gestione & \textbf{I}nformation Disclosure & Un attaccante può sfruttare le vulnerabilità scoperte nei software open source, come ad esempio OpenEMS, per compromettere i sistemi EMS delle aziende che li utilizzano. Alternativamente, l'attaccante può inserire codice malevolo nel progetto open source, che successivamente utilizzerà per condurre attacchi mirati all'esfiltrazione dei dati dell'azienda. & Attaccante esterno\\

%     D-01 & Disponibilità servizio & \textbf{D}enial of Service & L'attaccante riesce ad accedere alle cabine secondarie del DSO e si collega tramite uno switch tra il gateway e la RTU. Questa posizione privilegiata gli consente di intercettare il traffico di rete e identificare il server SCADA presente nel DMS. Una volta individuato il target, l'attaccante può lanciare un attacco DoS (magari utilizzando di un software come \textit{hping}) contro il servizio cloud che ospita il cluster Kubernetes, saturando il canale di trasmissione e causando latenze significative (\textit{Bottleneck}). Queste latenze compromettono gravemente l'invio tempestivo dei dati di telemetria, delle segnalazioni e dei comandi di controllo provenienti dallo SCADA Master, causando potenziali disfunzioni operative nella rete di distribuzione. \cite{threat-sotto-stazioni-paper} & Attaccante esterno\\ 
    

%     E-01 & Invio comandi & \textbf{E}levation of Privilege & Una volta ottenuto il controllo di un cluster Kubernetes del DMS sfruttando l'endpoint API utilizzato per l'invio delle segnalazioni da parte di RTU/IED, un attaccante può evitare di causare un singolo e vistoso disservizio, optando invece per sfruttare le capacità del DMS di controllare migliaia di dispositivi sul campo (RTU/IED) per lanciare un attacco distribuito e coordinato. L'attaccante può inviare comandi di apertura e chiusura degli interruttori o regolare la tensione dei trasformatori, causando oscillazioni di frequenza e tensione su tutta la rete di distribuzione, con effetti che possono propagarsi fino a perturbare la rete di alta tensione. & Attaccante esterno con aiuto da insider\\
   

% \hline


% \end{longtable}




% \section{Strategie di Mitigazione e Contromisure di Sicurezza}


% \renewcommand{\arraystretch}{1.5}
% \begin{longtable}{p{1.5cm}p{3.5cm}p{8cm}p{2.5cm}}
    
    
%     \caption{Mitigazione delle minacce} \\
    
%     \hline
%     \textbf{ID Minaccia }& \textbf{Descrizione minaccia} & \textbf{Mitigazione proposta} & \textbf{Categoria mitigazione}\\
%     \hline
%     \endfirsthead
    
%     \hline
%     \textbf{ID Minaccia }& \textbf{Descrizione minaccia} & \textbf{Mitigazione proposta} & \textbf{Categoria mitigazione}\\
%     \hline
%     \endhead

%     S-01 & Impersonificazione di SM & Durante la fase di acquisto dei dispositivi da utilizzare per centinaia di migliaia di dispositivi, il DSO deve imporre requisiti di sicurezza minimi obbligatori durante il bando di gare. L'utilizzo di pattern statistici e algoritmi di \textit{Anomaly Detection} da parte dell'AMI, può essere d'aiuto per valutare eventuali dispositivi compromessi. & Mitigare \\
    
%     T-01 & Manomissione dati in transito & L'utilizzo di crittografia e autenticazione End-to-End può rendere la manomissione dei dati difficile. L'EMS non si dovrebbe fidare ciecamente bensì deve tutelarsi incrociando dati da altri dispositivi utilizzando pattern statistici di correlazione. \cite{paper-threat-modelling} & Mitigazione \\

%     R-01 & Furto e/o modifica dei dati AMI & I dati una volta validati devono essere archiviati in un database \textit{WORM} (Write-Once, Read-Many) garantendo la non mutabilità del dato. Nessuna singola persona, a presciendere dal ruolo, dovrebbe avere i permessi per leggere, modificare e cancellare dati sensibili, sia di utenti sia aziendali. & Mitigazione.\\

%     I-01 & Inserimento di codice malevolo & Verificare attraverso code review l'utilizzo delle \textit{Best Practices} di sicurezza, implementative e le ulteriori librerie open-sorce utilizzate & Accettarlo \\

%     D-01 & Minare il funzionamento dello SCADA Master & La simulazione ha rivelato che il collo di bottiglia delle prestazioni durante un attacco DoS erano i router, la cui CPU veniva utilizzata al $100\%$, rendendoli irresponsabili sia tramite interfaccia web che CLI. Al contrario, il dispositivo di monitoraggio (che simula il server SCADA) ha mostrato un impatto minimo sull'utilizzo di CPU e RAM. Questo suggerisce che se il router è il punto debole, una parte significativa del traffico d'attacco può essere bloccata a quel livello, impedendo che raggiunga il server SCADA e potenzialmente l'intera rete attraverso l'impostazione di regole firewall appropriate \cite{threat-sotto-stazioni-paper}  &  Mitigarlo \\

%     E-01 & Attacco distribuito MT/BT & Nessun singolo processo deve avere la possibilità di controllare tutti i dispositivi di campo. I privilegi di comando devono essere segmentati geograficamente e/o per dispositivo, seguendo il modello RBAC. Inoltre per comandi ad alto impatto deve essere predisposto almeno una seconda approvazione da parte del personale. & Mitigazione\\
%     \hline
% \end{longtable}


% % PAPER DoS

% % Le fonti indicano che la mitigazione principale proposta per gli attacchi Denial of Service (DoS) è l'implementazione di configurazioni appropriate dei firewall.
% % Questa raccomandazione deriva direttamente dalle scoperte dello studio, che ha classificato gli attacchi DoS come i più critici tra le categorie di minacce identificate dal modello STRIDE. Ecco perché l'uso dei firewall è considerato cruciale:
% % •
% % Elevata criticità degli attacchi DoS: Gli attacchi DoS sono stati identificati come la categoria più critica di minacce, avendo il maggior numero di minacce ad alta priorità che necessitano di indagine (8 su 24 totali). La loro elevata probabilità di accadimento è dovuta alla possibilità di essere eseguiti con un singolo comando utilizzando strumenti disponibili pubblicamente come hping3, che non richiedono conoscenze specialistiche.
% % •
% % Impatto sulla comunicazione e sul controllo: La simulazione ha dimostrato che un traffico d'attacco di circa 14 Mbps con circa 30.000 pacchetti al secondo (PPS) è sufficiente a interrompere la comunicazione IEC104. Un attacco DoS riuscito può portare alla perdita dell'osservabilità e della controllabilità dell'intera smart grid.
% % •
% % Identificazione del "collo di bottiglia": I test nel modello di simulazione hanno rivelato che il collo di bottiglia delle prestazioni durante un attacco DoS erano i router, la cui CPU veniva utilizzata al 100%, rendendoli irresponsabili sia tramite interfaccia web che CLI. Questo è significativo perché, al contrario, il dispositivo di monitoraggio (che simulava il server SCADA) ha mostrato un impatto minimo sull'utilizzo di CPU e RAM.
% % •
% % Strategia di mitigazione: Poiché i router sono il punto debole in questo scenario, bloccare il traffico d'attacco a questo livello, ad esempio tramite firewall opportunamente configurati sui router stessi o a monte di essi, impedirebbe che la maggior parte del traffico dannoso raggiunga il server SCADA e potenzialmente l'intera rete. Se l'attacco riuscisse a bypassare le difese a livello di router e a raggiungere direttamente il server SCADA, avrebbe un potenziale maggiore di esaurire le risorse del server, causando la completa perdita di osservabilità e controllabilità dell'intera rete.
% % •
% % Applicazione pratica: Le aziende della rete elettrica in Norvegia stanno già utilizzando i risultati di questo studio per migliorare le proprie misure di sicurezza, specialmente contro gli attacchi DoS, proprio attraverso l'adozione di appropriate configurazioni dei firewall.
% % In sintesi, la mitigazione tramite configurazioni dei firewall si concentra sulla protezione dei punti di ingresso e degli elementi di rete critici (come i router) che fungono da "collo di bottiglia" per il traffico, prevenendo così che attacchi DoS facilmente eseguibili compromettano la funzionalità dell'intera smart grid.


% % \renewcommand{\arraystretch}{1.5}
% % \begin{longtable}{p{1.5cm}p{3.5cm}p{8cm}p{2.5cm}}
    
    
% %     \caption{Mitigazione delle minacce} \\
    
% %     \hline
% %     \textbf{ID Minaccia }& \textbf{Descrizione minaccia} & \textbf{Mitigazione proposta} & \textbf{Categoria mitigazione}\\
% %     \hline
% %     \endfirsthead
    
% %     \hline
% %     \textbf{ID Minaccia }& \textbf{Descrizione minaccia} & \textbf{Mitigazione proposta} & \textbf{Categoria mitigazione}\\
% %     \hline
% %     \endhead
    
% %     S-01 &  Service Account Impersonation& Principio di minimo privilegio: creare ruoli RBAC specifici e restrittivi per ogni Service Account evitando di inserire tutti i privilegi. Gestire le credenziali in modo opportuno utilizzando nei \textit{Secrets}, oggetti contenenti questi dati molto sensibili  & Mitigarla \\


% %     T-01 &  Attacco sfruttando vulnerabilità cloud provider & Google Cloud offre la possibilità di abilitare la funzionalità "Shielded Nodes". Garantisce un \textbf{avvio sicuro} dato che guarda se il software sia firmato digitalmente e non manomesso. Crea una hash dello stato di avvio e monitora che non cambi nel tempo.  & Mitigarla\\

% %     T-02 & Manomissione dati in transito & Utilizzo di cifratura asimmetrica sul canale di comunicazione (TLS) e codici di autenticazione e firma del messaggio per garantirne l'integrità (integrity). & Mitigarla \\

% %     R-01 & Ripudio di un comando inviato & Tutti i comandi, critici e non, devono essere firmati digitalmente dall'operatore e finire nel Audit Log per mantenerne traccia di tutte le operazioni rendendole sicure ed immutabili. Possono essere usati ledger blockchain o servizi che permettono solo log write-once & Eliminarla\\
    
    
% %     I-01 &  Regole firewall non adatte & Definire una solida regola di Network Policies in K8s per definire quali pod possono comunicare con quali altri pod, su quale IP, porta e protocollo magari definendo regole di \textit{Ingress} e \textit{Egress} & Mitigarla\\

% %     I-02 &  Configurazione errata firewall & L'utilizzo della rete Virtual Private Cloud (VPC) permette di sopperire ad un'errata configurazione del firewall  rendendo la comunicazioni privata tra i soli componenti della rete VPC. Eliminazione degli IP pubblici quando questi non sono strettamente necessari e accesso tramite rete VPN. & Mitigarla\\

% %     I-03 & Configurazione errata cloud storage & Applicazione di algoritmi per cifrare i dati presenti nel cloud store. Utilizzare policy di identificazione e ruolo con il principio del minimo privilegio. & Mitigarla\\

% %     D-01 & DDoS contro endpoint critici & Utilizzare servizi di anti-DDoS a livello di rete. Implementare il \textit{rate limiting} sugli endpoint API dei cluster K8s. Avere un piano nel caso questo evento dovesse comunque succedere. & Mitigare, Accettarla\\

% %     E-01 &  Non rispetto del Principio di minimo privilegio & Cercare di fare attenzione quando si inseriscono i privilegi ad un utente  & Mitigarla\\

% %     E-02 & Elevation of Privilege nel cluster K8s & Eseguire i container come utenti non-root ed utilizzare policy di sicurezza a livello di pod & Mitigarla \\
    
% %     \hline
% % \end{longtable}


% \section{Approcci alla Validazione delle Contromisure}

% La fase finale del ciclo di Threat Modeling, la validazione, è essenziale per garantire che le contromisure proposte siano efficaci e che la postura di sicurezza complessiva del sistema sia stata effettivamente migliorata. Sebbene un'implementazione e una validazione empirica completa dell'architettura proposta esulino dall'ambito di questa tesi, è possibile delineare un piano di validazione strutturato.

% Questo piano si baserebbe su una combinazione di revisioni di progettazione e test di sicurezza pratici, mirati a verificare le mitigazioni suggerite.

% \begin{enumerate}
%     \item \textbf{Revisione e Analisi del Rischio Residuo:}
%     \\Il primo passo consisterebbe in una revisione formale delle contromisure progettate per ogni minaccia. Per ciascuna, si dovrebbe valutare il rischio residuo, ovvero il livello di rischio che permane anche dopo l'implementazione della mitigazione. L'obiettivo è assicurarsi che tale rischio sia sceso al di sotto della soglia di accettabilità definita dagli stakeholder del sistema.
%     \item \textbf{Test di Sicurezza a Livello di Infrastruttura Cloud-Native:}
%     \\Per validare le contromisure legate alla configurazione di Kubernetes e dei container, si potrebbero eseguire le seguenti attività:
%     \begin{itemize}
%         \item Scansione delle Immagini dei Container: Utilizzare strumenti come Trivy o Clair per scansionare le immagini dei container (HES, DMS, etc.) alla ricerca di vulnerabilità note in librerie e dipendenze.
%         \item Analisi della Configurazione del Cluster (IaC Scanning): Impiegare strumenti come Kube-bench o Checkov per analizzare i file di configurazione di Kubernetes (Infrastructure as Code) e verificare che siano conformi alle best practice di sicurezza del CIS (Center for Internet Security).
%     \end{itemize}
%     \item \textbf{Test di Sicurezza a Livello Applicativo e di Rete:}
%     \\Per verificare le contromisure a livello di servizio, si potrebbero pianificare test più attivi:
%     \begin{itemize}
%         \item Penetration Testing mirato: Eseguire test di penetrazione focalizzati sui punti più critici emersi dall'analisi STRIDE. Ad esempio, si potrebbe tentare di sfruttare una debolezza nelle API esposte dal cluster EMS o di effettuare un attacco di Tampering sui dati scambiati tramite la VPN inter-cluster.
%         \item Creazione di Security Test Case: Sviluppare test automatici che verifichino specifiche contromisure. Ad esempio, un test potrebbe simulare una richiesta con un token di autenticazione invalido per assicurarsi che venga correttamente respinta, validando così una mitigazione contro lo Spoofing.
%     \end{itemize}
% \end{enumerate}


% L'esecuzione di queste attività di validazione fornirebbe un feedback concreto sull'efficacia delle strategie di mitigazione proposte e completerebbe il ciclo iterativo del Threat Modeling, trasformando l'analisi teorica in una base solida per un'implementazione sicura.
